%----------------------------------------------------------------------------------------
%	PACKAGES AND OTHER DOCUMENT CONFIGURATIONS
%----------------------------------------------------------------------------------------

\documentclass[letterpaper]{inzane_syllabus} % a4paper for A4

\usepackage{booktabs, colortbl, xcolor}
\usepackage{tabularx}
\usepackage{enumitem}
\usepackage{ltablex} 
\usepackage{multirow}

\setlist{nolistsep}

\usepackage{lscape}
\newcolumntype{r}{>{\hsize=0.9\hsize}X}
\newcolumntype{w}{>{\hsize=0.6\hsize}X}
\newcolumntype{m}{>{\hsize=.9\hsize}X}

\renewcommand{\familydefault}{\sfdefault}
\renewcommand{\arraystretch}{2.0}
%----------------------------------------------------------------------------------------
%	 PERSONAL INFORMATION
%----------------------------------------------------------------------------------------

\profilepic{logo.png} % Profile picture, if the height of the picture is less than that of the cirle, it will have a flat bottom. 


% To remove any of the following, you need to comment/delete the lines in the .cls file (c. line 186). Commenting/deleting the lines below will produce an error. 

%To add different lines, you will need to create the new command, e.g. \profPhone, in the .cls file c. line 76, and command to create the line in the side bar in the .cls file c. line 186

%\classname{AI for music\\and the arts} 
\classname{Computational\\creativity\\for music\\and the arts} 
\classnum{MUSIC 30} 

%%%%%%%%%%%%%%% PROF INFO
\profname{Carmine-Emanuele Cella}
\officehours{} 
\office{Center for New Music and Audio Technologies (CNMAT)}
\site{http://www.carminecella.com} 
\email{carmine.cella@berkeley.edu}

%%%%%%%%%%%%%%% COURSE  INFO
\prereq{Prereq: None}
\classdays{}
\classhours{}
\classloc{}

%%%%%%%%%%%%%%% LAB INFO
\labdays{}
\labhours{}
\labloc{Lab Space}

%%%%%%%%%%%%%%% TA INFO
\taAname{Alice}
\taAofficehours{Office Hrs: Tues \& Thurs 10-11a}
\taAoffice{MCZ 104}
% \taAemail{}
\taBname{James}
\taBofficehours{Office Hrs: Tues \& Thurs 3-4p}
\taBoffice{MCZ 104}
% \taBemail{}

%\about{Fish make up the largest group of vertebrates on the planet, easily outnumbering mammals, marsupials, birds, and reptiles combined. Not only are they abundant, but they've diversified into an extraordinary array of sizes, shapes, lifestyles, and habitats. You can find them in the coldest, deepest parts of the ocean, and in the hottest freshwater ponds in the desert. This course will explore fish diversity and their biology. } 


%---------------------------------------------------------------------------------------
%	 FAQs
%----------------------------------------------------------------------------------------
%to add more questions or remove this section, go to the .cls file and start with lines comment
%lines 226-250. Also comment out this section as well as line 152(ish), the command \makeSide

\qOne{Do we dissect real fish in this course?}
\aOne{Yes, we do actually dissect fish. If you know of any issues that may cause you difficulties during dissections, please notify your TA ASAP.}

\qTwo{What is a fish?}
\aTwo{No clue. When someone says `fish', we have a picture of a general fish of a general shape in our minds, but the truth is that `fish' doesn't have scientific meaning. Here's a funny video about that: \href{https://youtu.be/uhwcEvMJz1Y}{Youtube (hyperlink)}. }

\qThree{What is your favorite fish?}
\aThree{A lumpsucker. They are incredibly, adorably weird-looking.}

\qFour{What's the difference between plural `fish' and `fishes'?}
\aFour{`Fish' is the plural form when talking about two or more fish of the same species. `Fishes' is the plural when talking about two or more different species.}

%----------------------------------------------------------------------------------------

\begin{document}

%----------------------------------------------------------------------------------------
%	 DESCRIPTION
%----------------------------------------------------------------------------------------

\makeprofile % Print the sidebar

%----------------------------------------------------------------------------------------
%	 OVERVIEW
%----------------------------------------------------------------------------------------
\section{Overview}

During the first half of this course, we will work up the fish phylogeny, examining both extinct and extant lineages. In the second half, we'll dive deep into the specific systems fish have developed that allow them to dominate the aquatic world. We'll spend the last few weeks looking at their behavior, ecology, and some of the conservation efforts currently underway to help protect our fish populations. Throughout the semester, labs will help students connect what they have read and heard with what they can see and feel, reinforcing the material.


%----------------------------------------------------------------------------------------
%	 EXTRAS
%----------------------------------------------------------------------------------------

\vspace{0.5cm}
\section{Learning Objectives}

%use \begin{outline} or \begin{outline}[enumerate] to create a list with subitems. 
\begin{itemize}
\item Become familiar with the evolutionary history and taxonomic diversity of fishes
\item Improve your understanding of the basic physiological and behavioral adaptations of fishes
\item Gain skills regarding the dissection, collection, and preservation of fish specimens through laboratory work
\item Learn to critically review a paper and summarize it, as well as review and provide helpful criticism to your peers' work

\end{itemize}

%----------------------------------------------------------------------------------------
%	 READING MATERIAL
%----------------------------------------------------------------------------------------
\vspace{0.5cm} %I make liberal use of the \vspace{} command to partition and place sections just how I want them. Alter as you see fit. 
\section{Material}

{\color{myCOLOR} Required Texts}\\
Helfman, G.S., Collette, B.B., Facey, D.E., \& Bowen, B.W. \textit{The Diversity of Fishes: Biology, Evolution, and Ecology}. 2nd Edition. Wiley-Blackwell. 2009. ("DOF") \\

{\color{myCOLOR} Recommended Text}\\
Paxton, J.R. \& Eschmeyer, W.N. \textit{Encyclopedia of Fishes}. 2nd Edition. Harcourt Brace \& Co. 1998. \\

{\color{myCOLOR} Other}\\
Any required journal articles and book chapters will be provided on Canvas. 

%----------------------------------------------------------------------------------------
%	 GRADING SCHEME
%----------------------------------------------------------------------------------------
\vspace{0.5cm}
\section{Grading Scheme}

%below is the \twentyshort environment - a list with only two inputs. However, there is a \twenty environment, which creates a list with four inputs. You can find/alter details of that table in the .cls file c. lines 320. 
\begin{twentyshort}
	%\twentyitemshort{X\%}{Attendance/Participation}
	\twentyitemshort{15\%}{Lab Worksheets}
    \twentyitemshort{40\%}{Midterm Exams (20\% each)}
    \twentyitemshort{45\%}{Final Exam}
\end{twentyshort}

Grades will follow the standard scale: A = 89.5-100; B = 79.5-89.4; C = 69.5-79.4; D = 60-69.4; F  $<$60. Curving is at the discretion of the professor. 


%%%%%%%%%%%%%%%%%%%%%%%%%%%%%%%%%%%%%%%%%%%%%%%%%%%%%%%%%%%%%%%%%%%%%%%%%%%%%
%                SECOND PAGE
%%%%%%%%%%%%%%%%%%%%%%%%%%%%%%%%%%%%%%%%%%%%%%%%%%%%%%%%%%%%%%%%%%%%%%%%%%%%%

\newpage % Start a new page

\makeSide % Print the FAQ sidebar; To get rid of, simply comment out and uncomment \makeFullPage

% \makeFullPage

\vspace{0.5cm}
\section{Make-up Policy}

Make-up exams or assignments will only be allowed for students who have a substantiated excuse approved by the instructor \emph{before the due date}. Leaving a phone message or sending an e-mail without confirmation is not acceptable. Labs are mandatory.

\vspace{0.5cm}
\section{Diversity and Inclusivity Statement}

I consider this classroom to be a place where you will be treated with respect, and I welcome individuals of all ages, backgrounds, beliefs, ethnicities, genders, gender identities, gender expressions, national origins, religious affiliations, sexual orientations, ability - and other visible and non-visible differences. All members of this class are expected to contribute to a respectful, welcoming and inclusive environment for every other member of the class. 

\vspace{0.5cm}
\section{Accommodations for Students with Disabilities}

If you are a student with learning needs that require special accommodation, contact the Disabled Students' Program (https://dsp.berkeley.edu), as soon as possible, to make an appointment to discuss your special needs.  Please e-mail me in order to set up a time to discuss your learning needs.

\vspace{0.5cm}
\section{Academic Integrity}

Berkeley's honor code states "As a member of the UC Berkeley community, I act with honesty, integrity, and respect for others." The honor code is a cornerstone of our learning community and of this course. It is your responsibility to know and follow academic integrity policies. I will gladly answer any questions you have.\\

\vspace{0.5cm}
\section{Harassment and discrimination}

The University of California strives to prevent and respond to harassment and discrimination. Engaging in such behavior may result in removal from class or the University. If you are the subject of harassment or discrimination there are resources available to support you. Please contact the Confidential Care Advocate (sa.berkeley.edu/dean/confidential-care-advocate) for non-judgmental, caring assistance with options, rights and guidance through any process you may choose. Survivors of sexual violence may also want to view the following website: survivorsupport.berkeley.edu.
For more information about how the University responds to harassment and discrimination, please visit the Office for the Prevention of Harassment and Discrimination website: ophd.berkeley.edu.\\

%%%%%%%%%%%%%%%%%%%%%%%%%%%%%%%%%%%%%%%%%%%%%%%%%%%%%%%%%%%%%%%%%%%%%%%%%%%%%
%                COURSE SCHEDULE
%%%%%%%%%%%%%%%%%%%%%%%%%%%%%%%%%%%%%%%%%%%%%%%%%%%%%%%%%%%%%%%%%%%%%%%%%%%%%
\newpage
\makeFullPage
\section{Class Schedule}

\begin{center}
\begin{tabularx}{\textwidth}{p{2cm}p{8cm}p{9.5cm}} %change the width of the comments by changing these cm measurements. Add/substract columns by adding/deleting p{} sections. 
\arrayrulecolor{myCOLOR}\hline
%%%%%%%%%%%%%%%%%%%%%%%%%%%%%%%%%%%%%%%%%%% MODULE 1
\multicolumn{3}{l}{\textbf{\textcolor{myCOLOR}{\large MODULE 1: Representations }}} \\
\hline
% Week & Topic & Readings \\ \hline 
%%Alternatively, instead of Week #, you can do Class date for meeting
Week 1 & History of the Earth - Fish Remix & Friedman, M. \& Salland, L.C. (2012). Five hundred million years of extinction and recovery: A Phanerozoic survey of large-scale diversity patterns in fishes. \textit{Palaeontology}, 55(4):707-742 \\

& Stem \& Extant Agnathans \& Gnathostomes & DOF Ch. 11, pp. 169-179; Ch. 13, pp. 231-240  \\
& & Brazeau, M.D. \& Friedman, M. (2015). The origin and early phylogenetic history of jawed vertebrates. \textit{Nature}, 520(7548): 490-497.\\
\arrayrulecolor{maingray}\hline
Week 2 & Chondrichthyans I: Overview \& Sharks & DOF Ch. 11, pp. 197-200; Ch. 12, pp. 205-227\\

& Chondrichthyans II: Batoids \& Chimaeras & DOF Chapter 12, pp. 227-229 \\
\arrayrulecolor{maingray}\hline
Week 3 & Stem \& Extant Sarcopterygians & DOF Ch. 11, pp. 179-185; Ch. 13, pp. 242-248 \\

& Actinopts I: Overview & DOF Ch. 14 \& Ch. 15 \\
\arrayrulecolor{maingray}\hline
Week 4 & Actinopts II: Basal Actinopts \& Teleostei & DOF Ch. 11, pp. 185-197; Ch. 13, pp. 248-259, Ch. 14, pp. 261-266 \\

& Actinopts III: Otocephalan Fishes & DOF Ch. 14, pp. 267-275 \\

\arrayrulecolor{maingray}\hline
Week 5 & Actinopts IV: Freshwater Fishes & DOF Ch. 16, pp. 339-354; Ch. 18, pp. 410-414, 417-421 \\

& Actinopts V: Deep Sea Fishes & DOF Ch. 18, pp. 393-401 \\
& & Davis, M.P., Sparks, J.S., \& Smith, W. L. (2016). Repeated and widespread evolution of bioluminescence in marine fishes. \textit{PLOS One}.\\
\arrayrulecolor{maingray}\hline
Week 6 & Actinopts VI: Coral Reef Fishes & Bellwood, D.R. \& Wainwright, P.C. (2002). The History and Biogeography of Fishes on Coral Reefs. \textit{Coral Reef Fishes: Dynamics and Diversity in a Complex Ecosystem}, 5-32. \\

 & Actinopts VII: Pelagic Fishes & DOF Ch. 18, pp. 401-405 \\
 \arrayrulecolor{maingray}\hline
 Week 7 & Review & Module 1 \\
 &EXAM &  MIDTERM 1 \\
 
 \arrayrulecolor{myCOLOR}\hline
\multicolumn{2}{l}{\textbf{\textcolor{myCOLOR}{\large MODULE 2: What Makes a Fish }}} \\
\hline
 Week 8 & Respiration & DOF Ch. 5 \\
 
 & Cardiovascular Systems & DOF Ch. 4, pp. 45-48 \\
 \arrayrulecolor{maingray}\hline
Week 9 & Homeostasis & DOF Ch. 4, pp. 52; Ch. 7, pp. 101-105.\\

& Feeding Mechanisms & DOF Ch. 4, pp. 41-42; Ch. 8, pp. 119-126  \\
\arrayrulecolor{maingray}\hline
Week 10 & Sensory Systems & DOF Ch. 6 \\

&  Buoyancy & DOF Ch. 4, pp. 50-52  \& Ch. 5, pp. 68-70 \\
\arrayrulecolor{maingray}\hline
Week 11 &  Locomotion I - Undulatory Propulsion &  Webb, P.W. (1984). Form and function in fish swimming. \textit{Sci. Amer.}, 251(1): 72-83. \\
% & & Shadwick, R.E. (2005). How tunas and lamnid sharks swim: An evolutionary convergence. \textit{Amer. Sci.}, 93: 524-531. \\

& Locomotion II - Oscillatory Propulsion & Daniel, T.L. (1984). Unsteady Aspects of Aquatic Locomotion. \textit{Amer. Zoo.}, 24: 121-134.\\
\arrayrulecolor{maingray}\hline
Week 12 & Communication \& Reproduction  &  DOF Ch. 22, pp. 477-485 \\

& & DOF Ch. 21  \\

&Review & Module 2\\
\arrayrulecolor{maingray}\hline
Week 13 & EXAM & MIDTERM 2\\
&Holiday & Thanksgiving \\

\arrayrulecolor{myCOLOR}\hline

\multicolumn{2}{l}{\textbf{\textcolor{myCOLOR}{\large MODULE 3:  }}} \\
\hline
Week 14 & Symbiotic Relationships & DOF Ch. 22, 492-497 \\

& Behavior & DOF Ch. 23 \\
\arrayrulecolor{maingray}\hline
Week 15 & Ecology & DOF Ch. 25 \\

& Conservation Efforts & DOF Ch. 26 \\
\arrayrulecolor{myCOLOR}\hline
Week 16 & FINAL EXAM & Date \& Time \& Location \\ 
\hline 
\end{tabularx}
\end{center}

%%%%%%%%%%%%%%%%%%%%%%%%%%%%%%%%%%%%%%%%%%%%%%%%%%%%%%%%%%%%%%%%%%%%%%%%%%%%%
%                LAB SCHEDULE
%%%%%%%%%%%%%%%%%%%%%%%%%%%%%%%%%%%%%%%%%%%%%%%%%%%%%%%%%%%%%%%%%%%%%%%%%%%%%
\newpage
\section{Lab Schedule}

\begin{center}
\begin{tabularx}{\textwidth}{p{2cm}p{6.5cm}p{11cm}}
\arrayrulecolor{myCOLOR}\hline
Week 2 & Chondrichthyan Fishes & Students enjoy a two part lab: first, they examine specimens across the Chondrichthyan phylogeny; second, they dissect a small spiny dogfish shark. \\
\arrayrulecolor{maingray}\hline 
Week 3 & Harvard Natural History Museum & Students walk through the HMNH and the fossil collection, inspecting various fossil fishes. \\
\hline 
Week 4 & Basal Teleosts \& Otocephalan Fishes & Students explore specimens across the basal Teleost phylogeny. \\
\hline 
Week 5 & Freshwater \& Deep-Sea Fishes & Students explore specimens from a diverse group of fishes, and try to place each group in the broader phylogeny. \\
\hline 
Week 6 & Coral Reef \& Pelagic Fishes & Students explore specimens from a diverse group of fishes, and try to place each group in the broader phylogeny.\\
\hline 
Week 7 & No Lab & \\ 
\hline 
Week 8 & Internal Systems & Students dissect fish specimens, probing and examing key internal systems. \\
\hline 
Week 9 & Jaw Dissections & Students again dissect their fish specimens, taking apart and visualizing the jaws of their fish. \\
\hline 
Week 10 & Sensory Systems \& Buoyancy & Students again enjoy a two-part lab: first, examining a broad selection of specimens, comparing and contrasting sensory system apparatuses; and then conducting a series of small experiments to better understand the difficulties associated with buoyancy control in the water. \\
\hline 
Week 11 & Locomotion & Students dissect fish specimens, looking at muscular and structure of the body and fins. Students also participate in demonstrations designed to elucidate the concept of lift. \\
\hline 
Week 12 & Review Paper Projects & Students bring electronic devices and/or paper printouts of 2-3 paper choices, and will select peer reviewers. TAs will be available to assist students in choosing a paper and begin reviewing it. \\
\hline 
Week 13 & No Lab & \\
\hline 
Week 14 & No Lab & \\
\hline 
Week 15 & Final Exam Review Sessions & Review Paper Project Due \\
\arrayrulecolor{myCOLOR}\hline

\end{tabularx}
\end{center}

%----------------------------------------------------------------------------------------

\end{document} 



