%----------------------------------------------------------------------------------------
%	PACKAGES AND OTHER DOCUMENT CONFIGURATIONS
%----------------------------------------------------------------------------------------

\documentclass[letterpaper]{inzane_syllabus} % a4paper for A4

\usepackage{booktabs, colortbl, xcolor}
\usepackage{tabularx}
\usepackage{enumitem}
\usepackage{ltablex} 
\usepackage{multirow}

\setlist{nolistsep}

\usepackage{lscape}
\newcolumntype{r}{>{\hsize=0.9\hsize}X}
\newcolumntype{w}{>{\hsize=0.6\hsize}X}
\newcolumntype{m}{>{\hsize=.9\hsize}X}

\renewcommand{\familydefault}{\sfdefault}
\renewcommand{\arraystretch}{2.0}
%----------------------------------------------------------------------------------------
%	 PERSONAL INFORMATION
%----------------------------------------------------------------------------------------

\profilepic{logo.png} % Profile picture, if the height of the picture is less than that of the cirle, it will have a flat bottom. 


% To remove any of the following, you need to comment/delete the lines in the .cls file (c. line 186). Commenting/deleting the lines below will produce an error. 

%To add different lines, you will need to create the new command, e.g. \profPhone, in the .cls file c. line 76, and command to create the line in the side bar in the .cls file c. line 186

%\classname{AI for music\\and the arts} 
\classname{Computational\\creativity\\for music\\and the arts} 
\classnum{MUSIC 30} 

%%%%%%%%%%%%%%% PROF INFO
\profname{Carmine-Emanuele Cella}
\officehours{} 
\office{Center for New Music and Audio Technologies (CNMAT)}
\site{http://www.carminecella.com} 
\email{carmine.cella@berkeley.edu}

%%%%%%%%%%%%%%% COURSE  INFO
\prereq{Prereq: None}
\classdays{}
\classhours{}
\classloc{CNMAT (McEnerney Hall)}

%%%%%%%%%%%%%%% LAB INFO
\labdays{}
\labhours{}
\labloc{CNMAT (McEnerney Hall) - Laptop and headphones required}

%%%%%%%%%%%%%%% TA INFO
\taAname{Alice}
\taAofficehours{Office Hrs: Tues \& Thurs 10-11a}
\taAoffice{MCZ 104}
% \taAemail{}
\taBname{James}
\taBofficehours{Office Hrs: Tues \& Thurs 3-4p}
\taBoffice{MCZ 104}
% \taBemail{}

%\about{Fish make up the largest group of vertebrates on the planet, easily outnumbering mammals, marsupials, birds, and reptiles combined. Not only are they abundant, but they've diversified into an extraordinary array of sizes, shapes, lifestyles, and habitats. You can find them in the coldest, deepest parts of the ocean, and in the hottest freshwater ponds in the desert. This course will explore fish diversity and their biology. } 


%---------------------------------------------------------------------------------------
%	 FAQs
%----------------------------------------------------------------------------------------
%to add more questions or remove this section, go to the .cls file and start with lines comment
%lines 226-250. Also comment out this section as well as line 152(ish), the command \makeSide

\qOne{Do I need to know machine learning?}
\aOne{No. The essential tools of machine learning will be introduced in the course. However, some familiarity with linear algebra is recommended.}

\qTwo{Do I need to know Python programming?}
\aTwo{While it is not strictly necessary, it is recommended to know at least the basic structures of the language. Several labs will use Python programs and you will be asked to execute them. }

\qThree{Do I need to know Max programming?}
\aThree{No. All Max-based labs will use ready-to-use patches with a graphical interface.}

\qFour{How much musical knowledge is required?}
\aFour{Nothing more than an intuitive understanding of concepts such as melody or timbre. The motivation of the course, however, is to produce tools to create and transform sound so you should be at least interested in this.}

%----------------------------------------------------------------------------------------

\begin{document}

%----------------------------------------------------------------------------------------
%	 DESCRIPTION
%----------------------------------------------------------------------------------------

\makeprofile % Print the sidebar

%----------------------------------------------------------------------------------------
%	 OVERVIEW
%----------------------------------------------------------------------------------------
\section{Overview}

Computational creativity for music and the arts (MUS30) is..


%----------------------------------------------------------------------------------------
%	 EXTRAS
%----------------------------------------------------------------------------------------

\vspace{0.5cm}
\section{Learning Objectives}

%use \begin{outline} or \begin{outline}[enumerate] to create a list with subitems. 
\begin{itemize}
\item Understand...
\item Learn to critically review a paper and summarize it, as well as review and provide helpful criticism to your peers' work

\end{itemize}

%----------------------------------------------------------------------------------------
%	 READING MATERIAL
%----------------------------------------------------------------------------------------
\vspace{0.5cm} %I make liberal use of the \vspace{} command to partition and place sections just how I want them. Alter as you see fit. 
\section{Material}

{\color{myCOLOR} Required Text}\\
Cella, C. E. \textit{Creative computing for music and sound}. MIT Press, in preparation. ("CEL") \\

{\color{myCOLOR} Recommended Texts}\\
D. Benson, \textit{Music: a mathematical offering}, freely available on author's web page.  ("BEN")\\
A. Burkov, \textit{The Hundred-page machine learning book}, freely available on author’s web page, 2019. ("BUR")\\
A. G\'eron, \textit{Hands-on machine learning with Scikit-Learn \& TensorFlow}, O'Reilly, 2017.  ("GER")\\

{\color{myCOLOR} Other}\\
Music 30 will use Python/Anaconda (\url{https://www.anaconda.com}, link to external site) and Cycling’74 Max (\url{http://cycling74.com/}, link to external site) programming environments extensively during the labs. The open-ource audio editor Audacity (\url{https://www.audacityteam.org}, link to external site) will be also used during classes. Students must have access to a laptop computer with these software packages installed and must have headphones. Students may choose to purchase Max, or alternatively there are student authorization options.

Any required journal/conference articles and all the source code will be provided on \emph{bCourses}. 

%----------------------------------------------------------------------------------------
%	 GRADING SCHEME
%----------------------------------------------------------------------------------------
\vspace{0.5cm}
\section{Grading Scheme}

%below is the \twentyshort environment - a list with only two inputs. However, there is a \twenty environment, which creates a list with four inputs. You can find/alter details of that table in the .cls file c. lines 320. 
\begin{twentyshort}
	%\twentyitemshort{X\%}{Attendance/Participation}
	\twentyitemshort{20\%}{Assignments}
    \twentyitemshort{30\%}{Midterm Exam}
    \twentyitemshort{50\%}{Final Exam}
\end{twentyshort}

Grades will follow the standard scale: A = 89.5-100; B = 79.5-89.4; C = 69.5-79.4; D = 60-69.4; F  $<$60. Curving is at the discretion of the professor. 


%%%%%%%%%%%%%%%%%%%%%%%%%%%%%%%%%%%%%%%%%%%%%%%%%%%%%%%%%%%%%%%%%%%%%%%%%%%%%
%                SECOND PAGE
%%%%%%%%%%%%%%%%%%%%%%%%%%%%%%%%%%%%%%%%%%%%%%%%%%%%%%%%%%%%%%%%%%%%%%%%%%%%%

\newpage % Start a new page

\makeSide % Print the FAQ sidebar; To get rid of, simply comment out and uncomment \makeFullPage

% \makeFullPage

\vspace{0.5cm}
\section{Make-up Policy}

Make-up exams or assignments will only be allowed for students who have a substantiated excuse approved by the instructor \emph{before the due date}. Leaving a phone message or sending an e-mail without confirmation is not acceptable. Labs are mandatory.

\vspace{0.5cm}
\section{Diversity and Inclusivity Statement}

I consider this classroom to be a place where you will be treated with respect, and I welcome individuals of all ages, backgrounds, beliefs, ethnicities, genders, gender identities, gender expressions, national origins, religious affiliations, sexual orientations, ability - and other visible and non-visible differences. All members of this class are expected to contribute to a respectful, welcoming and inclusive environment for every other member of the class. 

\vspace{0.5cm}
\section{Accommodations for Students with Disabilities}

If you are a student with learning needs that require special accommodation, contact the Disabled Students' Program (https://dsp.berkeley.edu), as soon as possible, to make an appointment to discuss your special needs.  Please e-mail me in order to set up a time to discuss your learning needs.

\vspace{0.5cm}
\section{Academic Integrity}

Berkeley's honor code states "As a member of the UC Berkeley community, I act with honesty, integrity, and respect for others." The honor code is a cornerstone of our learning community and of this course. It is your responsibility to know and follow academic integrity policies. I will gladly answer any questions you have.\\

\vspace{0.5cm}
\section{Harassment and discrimination}

The University of California strives to prevent and respond to harassment and discrimination. Engaging in such behavior may result in removal from class or the University. If you are the subject of harassment or discrimination there are resources available to support you. Please contact the Confidential Care Advocate (sa.berkeley.edu/dean/confidential-care-advocate) for non-judgmental, caring assistance with options, rights and guidance through any process you may choose. Survivors of sexual violence may also want to view the following website: survivorsupport.berkeley.edu.
For more information about how the University responds to harassment and discrimination, please visit the Office for the Prevention of Harassment and Discrimination website: ophd.berkeley.edu.\\

%%%%%%%%%%%%%%%%%%%%%%%%%%%%%%%%%%%%%%%%%%%%%%%%%%%%%%%%%%%%%%%%%%%%%%%%%%%%%
%                COURSE SCHEDULE
%%%%%%%%%%%%%%%%%%%%%%%%%%%%%%%%%%%%%%%%%%%%%%%%%%%%%%%%%%%%%%%%%%%%%%%%%%%%%
\newpage
\makeFullPage
\section{Class Schedule}

\begin{center}
\begin{tabularx}{\textwidth}{p{2cm}p{8cm} @{\hskip 0.5cm} p{9.5cm}} %change the width of the comments by changing these cm measurements. Add/substract columns by adding/deleting p{} sections. 
\arrayrulecolor{myCOLOR}\hline
\hline 
\hline 

%%%%%%%%%%%%%%%%%%%%%%%%%%%%%%%%%%%%%%%%%%% MODULE 1
\multicolumn{3}{l}{\textbf{\textcolor{myCOLOR}{\large MODULE 1: Foundations }}} \\
\hline
  \# & Topic & Readings \\ \hline 
%%Alternatively, instead of Week #, you can do Class date for meeting
Week 1 &
Computational creativity is not \ldots creative: the four \emph{P}s of creativity & F. Carnovalini and A. Rod\`a,Computational Creativity and Music Generation Systems: An Introduction to the State of the Art. Front. Artif. Intell. 3:14, 2020  \\

& Creative artefacts vs assisted creation: dualities in modelling creativity &  R. L. DE M\`antaras, Artificial Intelligence and the Arts: Toward Computational Creativity, in The Next Step: Exponential Life, 2017\\
\arrayrulecolor{maingray}\hline

Week 2 & Three introductory views on sound: physical, perceptual and cultural & [BEN, ch. 1.1-1.7]\\

& Introduction to musical timbre and digital signals & [BEN, ch. 7.1-7.6, appendix M]\\
\arrayrulecolor{maingray}\hline

Week 3 & The five pillars of creative computing (I): probabilities &  \\

& \emph{Lab (Python)}: Markov models for text and music generation & \\
\arrayrulecolor{maingray}\hline

Week 4 & The five pillars of creative computing (II): projective spaces & C. E. Cella, A geometric interpretations of signals, 2015, available on www.carminecella.com\\

& \emph{Lab (Python)}: transforming sounds and images with convolutional maps & C. E. Cella, On room impulse response measurements with sine sweeps, 2017, available on www.carminecella.com \\
& & \\ 
\arrayrulecolor{maingray}\hline

Week 5 & The five pillars of creative computing (III): unsupervised and supervised statistical learning & [BUR, ch. 1]\\

& & C. E. Cella, Logistic regression and artificial neural networks, 2015, available on www.carminecella.com \\

& \emph{Lab (Python)}: classification of musical timbres & [BUR, ch. 2.7] \\

& & V. Lonstalen, C. E. Cella, Deep convolutional networks on the pitch spiral for musical instrument recognition, ISMIR 2016, New York, USA\\

\arrayrulecolor{maingray}\hline

Week 6 & The five pillars of creative computing (IV): logical rules and generative grammars&  \\

 &  \emph{Lab (Python)}: L-systems for natural patterns and melodic generation & \\
 \arrayrulecolor{maingray}\hline
 
 Week 7 & The five pillars of creative computing (V): optimisation &  \\
 
 & \emph{Lab (Max)}: computer-assisted orchestration with Orchidea &  \\
 \arrayrulecolor{maingray}\hline

 Week 8 & Review & Module 1 \\
 &EXAM &  MIDTERM \\
& & \\ 
 \arrayrulecolor{myCOLOR}\hline
\multicolumn{2}{l}{\textbf{\textcolor{myCOLOR}{\large MODULE 2: Transformations }}} \\
\hline

Week 9 & Modelling time and timbre in music & H. C. Crayencour, C. E. Cella, Learning, probability and logic: towards a unified approach for content-based Music Information Retrieval, Frontiers in Digital Humanities, April 2019 \\

& \emph{Lab (Max and Python)}: granular synthesis and \emph{AudioGuide} &  \\
\arrayrulecolor{maingray}\hline

Week 10 & Modeling musical style (I): projections &  \\

&   \emph{Lab (Max/MSP)}: spectral freeze and cross-synthesis, a prelude to musical style transfer &  \\
\arrayrulecolor{maingray}\hline

Week 11 & Modeling musical style (II): probabilities and unsupervised learning together & C. E. Cella, Sound-types: a new framework for symbolic sound analysis and synthesis, ICMC 2011, Huddersfield, United Kingdom  \\

&\emph{Lab (Python)}: sound-types, a further step towards musical style transfer & C. E. Cella and J.J. Burred, Advanced sound hybridizations by means of the theory of sound-types, ICMC 2013, Perth, Australia \\

\arrayrulecolor{maingray}\hline
Week 12 & Modeling musical style (III): supervised learning with deep neural networks &  L. Gabrielli, C. E. Cella, F. Vespertini, D. Droghini, E. Principi and S. Squartini, Deep Learning for Timbre Modification and Transfer: an Evaluation Study, AES 144th, 2018, Milan, Italy \\
&\emph{ Lab (Python)}: an algorithm for universal musical style transfer &   Noam Mor, Lior Wolf, Adam Polyak, Yaniv Taigman, A universal music translation network, ICLR 2019 \\

& & \\ 
\arrayrulecolor{myCOLOR}\hline
\multicolumn{2}{l}{\textbf{\textcolor{myCOLOR}{\large MODULE 3: Connections  }}} \\
\hline
\arrayrulecolor{maingray}\hline

Week 13 & Extending the techniques to other arts: image style transfer & \\
&Evaluation of creative outcomes &  \\
\arrayrulecolor{maingray}\hline

Week 14 & On the aesthetics of computational creativity &  \\

& Social impact of assisted creation &  \\
\arrayrulecolor{maingray}\hline

Week 15 & Review & Modules 2 and 3\\

& EXAM & FINAL \\
\arrayrulecolor{myCOLOR}\hline
\hline 
\hline 
\hline 
\end{tabularx}
\end{center}


\end{document} 



